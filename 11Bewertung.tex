%%%%%%%%%%%%%%%%%%%%%%%%%%%%%%%%%%%%%%%%%%%%%%%%%%%%%%%%%%%%%%%%%
%        Contents: Bachelorarbeit, HS Fulda        %
%                          31.08.2022                        %
%---------------------------------------------------------%
%                         Bewertung.tex                     %
%                        by Fangfang Tan                    %
%         fangfang.tan@informatik.hs-fulda.de      %
%%%%%%%%%%%%%%%%%%%%%%%%%%%%%%%%%%%%%%%%%%%%%%%%%%%%%%%%%%%%%%%%%

\chapter{Bewertung der Technologien} \label{EV}

Eines der Ziele dieser Bachelorarbeit ist es, die Vor- und Nachteile von SAP AppGyver, SAP Fiori Elements und SAPUI5 zu identifizieren und herauszufinden, welche Technologie für welche Implementierungsszenarien geeignet ist. In Kapitel 3 und 4 wurden die technische Umsetzung und weitere Funktionen beschrieben. In diesem Kapitel werden nun zwei Bewertungsmatrizen definiert, diese mit Bewertungen gefüllt und anschließend die Bewertungsergebnisse interpretiert und diskutiert.

Die Bewertungsmatrizen beziehen sich zum einen auf die Bewertung der Funktionen des jeweiligen Tools, sowie die Möglichkeit zur Abbildung der definierten Anforderungen. Eine zweite Matrix wird dann für die Entwickler-Perspektive angefertigt.

\section{Bewertungsmatrixen definieren}
Die Bewertungsmatrizen besitzen die Form einer Tabelle \cite{wi:ma} und jeweils einen Tabellenkopf mit den Spaltenbezeichnungen, eine Tabellenvorspalte mit der Beschreibung der Funktionen und die Bewertung als eigentlichen Tabelleninhalt. Der Tabelleninhalt zeigt so die Zusammenhänge zwischen den Zeilen und Spalten. \cite{wi:ta} 

\subsection{Bewertungsmatrix zur Implementierung der Funktionalität}
In Kapitel 3 und 4 wurde die technische Umsetzung der Anforderungen oder eventuell vorhandene Umsetzungsansätze (Kapitel 4) im Detail beschrieben.    Dabei lassen sich diese grundsätzlich in zwei Bereiche aufteilen: Backend- und Frontend-Funktionalität. 
Die Backend-Funktionen umfassen:
\begin{itemize}[noitemsep]
\item Möglichkeiten zur Datenmodellierung.
\item Möglichkeiten zur Datenerstellung, -speicherung, -bearbeitung und löschung.
\item Möglichkeiten zur zentralen Datenbereitstellung.
\end{itemize}
Es wurde bereits darauf hingewiesen, dass nicht alle Tools über diese Art von Backend-Funktionalitäten verfügen.
In Kapitel 3 und Kapitel 4 wurden weiterhin diverse Frontend-Funktionen implementiert und untersucht. Tabelle 5.1 zeigt die volle Bewertungsmatrix der Funktionen.

\begin{table}[htbp]\scriptsize
    \centering
    \setlength{\leftmargini}{0.4cm}
    \resizebox{\linewidth}{!}{
    \begin{tabular}{|>{\columncolor{mygrey2}}  p{3.5cm}  | l | l | l |}
        \hline
        \rowcolor{mygrey2} \diagbox{Funktionen}{Tools} & Fiori Elements & AppGyver & SAPUI5  \\
        \hline
        Datenmodellierung & & &  \\
        \hline
        \makecell[l]{Datenerstellung, \\ -speicherung, \\ -bearbeitung \\ und -löschung} &  &  &  \\
        \hline
        Servicebereitstellung & & &  \\
        \hline
        \makecell[l]{Listansicht zur Anzeige \\ aller Produkte} & & & \\
        \hline
        \makecell[l]{Einzelansicht für ein \\ Produkt} & & &  \\
        \hline
        \makecell[l]{Maske zum Pflegen eines \\ einzelnen Produkts} & & &  \\
        \hline
        \makecell[l]{Integration von \\ Suchfiltern} &  &  &  \\
        \hline
        \makecell[l]{Integration von Bild und \\ PDF-Dateien} & & &  \\
        \hline
        \makecell[l]{Integration einer Barcode \\ Scanner Funktionen} & & & \\
        \hline
        \makecell[l]{Nutzung mobiler \\ Funktionen} & & &  \\
        \hline
        \makecell[l]{Deployment für \\ unterschiedliche Endgeräte} & & & \\
        \hline
        \makecell[l]{Freie \\ Gestaltungsmöglichkeiten} & & &  \\
        \hline
    \end{tabular}
    }
  \caption{Bewertungsmatrix zur Implementierung der Funktionalität} 
\end{table}

Der Inhalt der Matrix wird mit Punktwerten belegt. Die Bewertungsmatrix umfasst zwei grundsätzliche Bewertungskriterien: die Umsetzbarkeit der zu bewertenden Funktionalitäten und den Implementierungsaufwand. Die Erfüllung der Kriterien wird auf einer Skala von 0 bis 3 eingestuft. Ein Wert von 0 bedeutet, dass die bewertete Funktion nicht umgesetzt werden kann. Der Aufwand wird im Rahmen dieser Bachelorarbeit wie folgt definiert: Der Aufwand wird gleichgesetzt mit der zeitlichen Dauer der Umsetzung, gemessen in Personentagen. Die Dauer umfasst dabei lediglich die reine technische Implementierungszeit und keine weiteren Aktivitäten, wie Konzeption, Testen oder Dokumentation. Die zeitliche Bewertung folgt dabei dem Wissenstand der Autorin in der jeweiligen Rolle:

\begin{itemize}[noitemsep]
\item Citizen Developer: SAP AppGyver und SAP Fiori Elements.
\item Pro Developer: SAPUI5.
\end{itemize}

Ein Wert von 1 bedeutet, dass die Funktion zwar implementiert werden kann, aber nur mit einem hohen Aufwand, also einer Umsetzungsdauer von mehr als 1 Personentag. Ein Wert von 2 bedeutet, dass die Umsetzung zwischen 0,5 bis 1 Personentage in Anspruch nimmt. Ein Wert von 3 bedeutet, dass die Funktion schnell umgesetzt werden kann, in wenigen Minuten bis zu einem halben Personentag.

\begin{table}[htbp]
    \centering
    \begin{tabular}{| c |}
        \hline
        \rowcolor{mygrey2} Erfüllungskriterien: 0 bis 3  \\
        \hline
        \makecell[l]{0 = nicht umsetzbar \\ 1 = mehr als 1 PT \\ 2 = 0,5 bis 1 PT \\ 3 = weniger als 0,5 PT}  \\
        \hline
    \end{tabular}
 \caption{Erfüllung Kriterien der Bewertungsmatrix} 
\end{table}

\subsection{Bewertungsmatrix für die Entwicklerperspektive}

Für die Entwicklerperspektive wird eine Matrix zur Bewertung der Entwicklungsumgebungen und Entwicklungstools für die drei Technologien definiert. Folgende Tabelle liefert eine Auflistung aller Bewertungskriterien: 
\begin{table}[htbp]\scriptsize
    \centering
    \setlength{\leftmargini}{0.4cm}
    \resizebox{\linewidth}{!}{
    \begin{tabular}{|>{\columncolor{mygrey2}}  p{3.5cm}  | l | l | l |}
        \hline
        \rowcolor{mygrey2} \diagbox{\makecell[l]{Entwickle-\\perspektive}}{Tools} & Fiori Elements & AppGyver & SAPUI5  \\
        \hline
        \makecell[l]{Notwendiges technisches \\ Verständnis} & & &  \\
        \hline
        \makecell[l]{Bedienbarkeit der \\ Entwicklungsumgebung} &  &  &  \\
        \hline
        \makecell[l]{Vollständigkeit der \\ Funktionen der \\ Entwicklungsumgebung} & & &  \\
        \hline
        \makecell[l]{Spezialisierung der \\ Entwicklungsumgebung} & & & \\
        \hline
        \makecell[l]{Einrichtungsaufwand der \\ Entwicklungsumgebung} & & &  \\
        \hline
        \makecell[l]{Plattformunabhängigkeit  \\ der Entwicklungsumgebung} & & &  \\
        \hline
        \makecell[l]{Stabilität der \\ Entwicklungsumgebung} &  &  &  \\
        \hline
        \makecell[l]{Dokumentation für \\ Programmierer} & & &  \\
        \hline
        Debugger & & & \\
        \hline
        Versionsverwaltung & & &  \\
        \hline
    \end{tabular}
    }
  \caption{Bewertungsmatrix aus der Entwickler-Perspektive} 
\end{table}

Der Inhalt der Matrix wird auch hier mit Punktwerten belegt, wobei jede Tabellenvorspalte eigene Bewertungskriterien umfasst. Die Erfüllung der Kriterien wird auf einer Skala von 1 (grundsätzlich negativ) bis 3 (grundsätzlich positiv) eingestuft. Die Erfüllungskriterien werden in Tabelle 5.4 detailliert vorgestellt.

\begin{table}[htbp]\small
    \centering
    \setlength{\leftmargini}{0.4cm}
    \begin{tabular}{|>{\columncolor{mygrey2}}  p{4cm}  | p{3cm} | p{3cm} | p{3cm} |}
        \hline
        \rowcolor{mygrey2} \diagbox{\makecell[l]{Entwickle-\\perspektive}}{\makecell[r]{Erfüllungs-\\kriterien}} & 1 & 2 & 3  \\
        \hline
        \makecell[l]{Notwendiges technisches \\ Verständnis} & Programmierausbildung notwendig  & Grundlegende IT-Prinzipien notwendig & Kein IT-Verständnis notwendig  \\
        \hline
        \makecell[l]{Bedienbarkeit der \\ Entwicklungsumgebung} & Komplex und nur mit mehrfacher Anleitung/nach Schulung bedienbar  &  Komplexer, aber nach erstmaliger Nutzung verständlich & Einfach und selbsterklärend   \\
        \hline
        \makecell[l]{Vollständigkeit der \\ Funktionen der \\ Entwicklungsumgebung} & Wichtige Funktionen fehlen & Funktionen fehlen, können aber über Erweiterungen installiert werden & Alle notwendigen Funktionen integriert \\
        \hline
        \makecell[l]{Spezialisierung der \\ Entwicklungsumgebung} & exklusiv und nicht für andere Dinge nutzbar & Unterstützung weiterer Programmiersprachen und Anwendungstypen & Offen und Unterstützung vieler Anwendungsfälle  \\
        \hline
        \makecell[l]{Einrichtungsaufwand der \\ Entwicklungsumgebung} & Hoch (Manuell, mehrstufig, Dauer > 1 Stunde) & Mittel (Manuell, Konfiguration notwendig, Dauer > 10 Minuten < 1 Stunde) & Niedrig (Automatisch, nur wenig Konfiguration, Dauer < 10 Minuten) \\
        \hline
        \makecell[l]{Plattformunabhängigkeit  \\ der Entwicklungsumgebung} & Spezifisch & Teilweise abhängig & Systemunabhängig \\
        \hline
        \makecell[l]{Stabilität der \\ Entwicklungsumgebung} & Instabil inklusive Datenverlust  & Gelegentliche Instabilität &  Stabil \\
        \hline
        \makecell[l]{Dokumentation für \\ Programmierer} & Nicht vorhanden & Unvollständig & Vollständig \\
        \hline
        Debugger & Nicht vorhanden & Komplexe Nutzung & Einfache Nutzung \\
        \hline
        Versionsverwaltung & Nicht integriert & Integrierbar & Integriert  \\
        \hline
    \end{tabular}
  \caption{Bewertungsmatrix aus der Entwickler-Perspektive} 
\end{table}

todo: Tabelle 5.4 Erfüllungskriterien aus der Entwickler-Perspektive

\section{Durchführung der Bewertung}

\subsection{Bewertung zur Implementierung der Funktionalität}