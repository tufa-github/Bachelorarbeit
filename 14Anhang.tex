%%%%%%%%%%%%%%%%%%%%%%%%%%%%%%%%%%%%%%%%%%%%%%%%%%%%%%%%%%%%%%%%%
%        Contents: Bachelorarbeit, HS Fulda        %
%                          31.08.2022                        %
%---------------------------------------------------------%
%                           Anhang.tex                     %
%                        by Fangfang Tan                    %
%         fangfang.tan@informatik.hs-fulda.de      %
%%%%%%%%%%%%%%%%%%%%%%%%%%%%%%%%%%%%%%%%%%%%%%%%%%%%%%%%%%%%%%%%%

\chapter{Quellencode der SAPUI5 Anwendung - Products Information}

\begin{mdframed}[backgroundcolor=mygrey2, leftmargin=0.5cm, hidealllines=true, innerleftmargin=3pt, innerrightmargin=0cm, innertopmargin=0cm, innerbottommargin=-3cm, splitbottomskip=0]
\begin{lstlisting}[language=json, caption={Manifest Datei des UI5 Projekts in JSON Fromat}]
{
	"_version": "1.12.0",

	"sap.app": {
		"id": "tff.Products",
		"type": "application",
		"i18n": "i18n/i18n.properties",
		"title": "{{appTitle}}",
		"description": "{{appDescription}}",
		"applicationVersion": {
			"version": "1.0.0"
		}
	},

	"sap.ui": {
		"technology": "UI5",
		"icons": {

		},
		"deviceTypes": {
			"desktop": true,
			"tablet": true,
			"phone": true
		}
	},

	"sap.ui5": {
		"rootView": {
			"viewName": "tff.Products.view.App",
			"type": "XML",
			"async": true,
			"id": "app"
		},

		"dependencies": {
			"minUI5Version": "1.105.3",
			"libs": {
				"sap.ui.core": {},
				"sap.ui.layout": {},
				"sap.ui.unified": {},
				"sap.m": {}
			}
		},

		"handleValidation": true,

		"contentDensities": {
			"compact": true,
			"cozy": true
		},

		"models": {
			"i18n": {
				"type": "sap.ui.model.resource.ResourceModel",
				"settings": {
					"bundleName": "tff.Products.i18n.i18n"
				}
			}
		},

		"routing": {
			"config": {
				"routerClass": "sap.m.routing.Router",
				"viewType": "XML",
				"viewPath": "tff.Products.view",
				"controlId": "app",
				"controlAggregation": "pages",
				"async": true
			},
			"routes": [
				{
					"pattern": "",
					"name": "main",
					"target": "main"
				},
				{
					"pattern": "detail/{path}",
					"name": "detail",
					"target": "detail"
				},
				{
					"pattern": "barcode",
					"name": "barcode",
					"target": "barcode"
				}
			],
			"targets": {
				"main": {
					"viewId": "main",
					"viewName": "Main"
				},
				"detail": {
					"viewId": "detail",
					"viewName": "Detail"
				},
				"barcode": {
					"viewId": "barcode",
					"viewName": "Barcode"
				}
			}
		}
	}
}
\end{lstlisting} 
\end{mdframed}

\begin{mdframed}[backgroundcolor=mygrey2, leftmargin=0.5cm, hidealllines=true, innerleftmargin=3pt, innerrightmargin=0cm, innertopmargin=0cm, innerbottommargin=-3cm, splitbottomskip=0]
\begin{lstlisting}[language=XML,  caption={Main View der Anwendung in XML Format}]
<mvc:View controllerName="tff.Products.controller.Main" displayBlock="true" xmlns="sap.m" xmlns:mvc="sap.ui.core.mvc">
	<Page title="Products Information">
		<List mode="Delete" delete="onDeletePress" growing="true" growingThreshold="8" items="{path:'/Products'}" id="productsList">
			<headerToolbar>
				<OverflowToolbar width="100%">
					<Title text="The list of the Products" />
					<ToolbarSpacer />
					<Button icon ="sap-icon://camera" text="Scan" press="onScanNavi"/>
					<Button icon="sap-icon://add" text="Create" ariaHasPopup="Dialog" press="pressAddInList" />
					<SearchField width="200px" placeholder="SearchProduct" search="mySearchProduct" />
				</OverflowToolbar>
			</headerToolbar>
			<items>
				<StandardListItem title="{title}" type="Navigation" info="{price}" press="onItemPress"/>
			</items>
		</List>
	</Page>
</mvc:View>
\end{lstlisting}
\end{mdframed}

\begin{mdframed}[backgroundcolor=mygrey2, leftmargin=0.5cm, hidealllines=true, innerleftmargin=3pt, innerrightmargin=0cm, innertopmargin=0cm, innerbottommargin=-3cm, splitbottomskip=0]
\begin{lstlisting}[language=XML,  caption={Detail View der Anwendung in XML Format}]
<mvc:View
	controllerName="tff.Products.controller.Detail"
	displayBlock="true"
	xmlns="sap.m"
	xmlns:mvc="sap.ui.core.mvc">
   <Page title="Product Detail Page" showNavButton="true" navButtonPress="onNaviBack">
    <VBox>  
            <Label text="Title" />
            <Input value="{title}" width="1000px" />
            <Label text="Description" />
            <Input value="{description}" width="1000px" />
            <Label text="Material Number" />
            <Input value="{materialNumber}" width="1000px" />
            <Label text="Price" />
            <Input value="{price}" width="1000px" />
            <Label text="Stock" />
            <Input value="{stock}" width="1000px" />  
     </VBox>   
    <Image src="./images/seatImage.jpg" width="200px" height="150px"/>
    <Image src="https://prod.pictures.autoscout24.net/listing-images/0693be87-aaad-4dc8-99c5-a69b3bd48b81_1934e390-16f2-4e47-b248-02e9cc9b746b.jpg"
           width="200px" height="150px"/>
    <PDFViewer source="https://www.uni-muenster.de/imperia/md/content/zns/dokumente/beispielexpos___kowi_empirisch.pdf"/>
    </Page>
</mvc:View>
\end{lstlisting}
\end{mdframed}

\begin{mdframed}[backgroundcolor=mygrey2, leftmargin=0.5cm, hidealllines=true, innerleftmargin=3pt, innerrightmargin=0cm, innertopmargin=0cm, innerbottommargin=-3cm, splitbottomskip=0]
\begin{lstlisting}[language=XML,  caption={CreateProductDialog fragment der Anwendung in XML Format}]
<core:FragmentDefinition xmlns="sap.m" xmlns:core="sap.ui.core">
    <Dialog id="myProductView" title="" class="sapUiResponsivePadding--header sapUiResponsivePadding--footer" initialFocus="email">
        <content>
            <VBox class="sapUiMediumMargin">
                <Label text="Title" />
                <Input value="{listInputModel>/recipient/title}" width="200px" />
                <Label text="Material Number" />
                <Input value="{listInputModel>/recipient/materialNumber}" width="200px" />
                <Label text="Description" />
                <Input value="{listInputModel>/recipient/description}" width="200px" />
                <Label text="Price" />
                <Input value="{listInputModel>/recipient/price}" width="200px" />
                <Label text="Stock" />
                <Input value="{listInputModel>/recipient/stock}" width="200px" />
            </VBox>
        </content>
        <beginButton>
            <Button text="Cancel" ariaHasPopup="Dialog" press="closeDialog" />
        </beginButton>
        <endButton>
            <Button text="Add" ariaHasPopup="Dialog" press="pressAddInProductsList" />
        </endButton>
    </Dialog>
</core:FragmentDefinition>
\end{lstlisting}
\end{mdframed}

\begin{mdframed}[backgroundcolor=mygrey2, leftmargin=0.5cm, hidealllines=true, innerleftmargin=3pt, innerrightmargin=0cm, innertopmargin=0cm, innerbottommargin=-3cm, splitbottomskip=0]
\begin{lstlisting}[language=XML,  caption={Barcode View der Anwendung in XML Format}]
<mvc:View controllerName="tff.Products.controller.Barcode" displayBlock="true" xmlns:ndc="sap.ndc" xmlns="sap.m" xmlns:mvc="sap.ui.core.mvc">
<VBox class="sapUiSmallMargin">
		<ndc:BarcodeScannerButton
			id="button"
			scanSuccess="onScanSuccess"
			scanFail="onScanError"
			dialogTitle="Barcode Scanner Button Sample"/>
		<HBox class="sapUiTinyMarginTop">
			<Label text="Scan Result:"/>
			<Text id="result" text="" class="sapUiTinyMarginBegin"/>
		</HBox>
	</VBox>
</mvc:View>
\end{lstlisting}
\end{mdframed}

\begin{mdframed}[backgroundcolor=mygrey2, leftmargin=0.5cm, hidealllines=true, innerleftmargin=3pt, innerrightmargin=0cm, innertopmargin=0cm, innerbottommargin=-3cm, splitbottomskip=0]
\begin{lstlisting}[language=XML,  caption={App view der Anwendung in XML Format}]
<mvc:View
	controllerName="tff.Products.controller.App"
	displayBlock="true"
	xmlns="sap.m"
	xmlns:mvc="sap.ui.core.mvc">
	<App id="app" />
</mvc:View>
\end{lstlisting}
\end{mdframed}

\begin{mdframed}[backgroundcolor=mygrey2, leftmargin=0.5cm, hidealllines=true, innerleftmargin=3pt, innerrightmargin=0cm, innertopmargin=0cm, innerbottommargin=-3cm, splitbottomskip=0]
\begin{lstlisting}[emph={event, UI5Event, listItem, detail, path}, caption=Main Controller der Anwendung]
import MessageBox from "sap/m/MessageBox";
import BaseController from "./BaseController";
import formatter from "../model/formatter";
import Dialog from "sap/m/Dialog";
import Contextf from "sap/ui/model/odata/v4/Context";
import Fragment from "sap/ui/core/Fragment";
import DateTimeOffset from "sap/ui/model/odata/type/DateTimeOffset";
import UI5Event from "sap/ui/base/Event";
import ColumnListItem from "sap/m/ColumnListItem";
import Context from "sap/ui/model/odata/v4/Context";
import ODataListBinding from "sap/ui/model/odata/v4/ODataListBinding";
import Filter from "sap/ui/model/Filter";
import FilterOperator from "sap/ui/model/FilterOperator";

/**
 * @namespace tff.Products.controller
 */
export default class Main extends BaseController {
  private formatter = formatter;
  private addDialog: Dialog;

  public closeDialog() {
    this.addDialog.close();
  }

  public pressAddInProductsList() {
    var title = this.getView()
      .getModel("listInputModel")
      .getProperty("/recipient/title");
    var id = this.getView()
      .getModel("listInputModel")
      .getProperty("/recipient/id");
    var description = this.getView()
      .getModel("listInputModel")
      .getProperty("/recipient/description");
    var materialNumber = this.getView()
      .getModel("listInputModel")
      .getProperty("/recipient/materialNumber");
    var price = this.getView()
      .getModel("listInputModel")
      .getProperty("/recipient/price");
    var stock = this.getView()
      .getModel("listInputModel")
      .getProperty("/recipient/stock");

    const context = (<ODataListBinding>(
      this.getView().byId("productsList").getBinding("items")
    )).create({
      title: title,
      id: id,
      description: description,
      materialNumber: parseInt(materialNumber),
      price: price,
      stock: stock,
    });
    this.closeDialog();
  }

  public async pressAddInList() {
    var title = this.getView()
      .getModel("listInputModel")
      .getProperty("/recipient/title");
    var id = this.getView()
      .getModel("listInputModel")
      .getProperty("/recipient/id");
    var description = this.getView()
      .getModel("listInputModel")
      .getProperty("/recipient/description");
    var materialNumber = this.getView()
      .getModel("listInputModel")
      .getProperty("/recipient/materialNumber");
    var price = this.getView()
      .getModel("listInputModel")
      .getProperty("/recipient/price");
    var stock = this.getView()
      .getModel("listInputModel")
      .getProperty("/recipient/stock");

    if (!this.addDialog) {
      this.addDialog = <Dialog>await Fragment.load({
		name: "tff.Products.view.createProductDialog",
        controller: this,
      });
      this.getView().addDependent(this.addDialog);
    }
    this.addDialog.open();
  }

  public onDeletePress(event: UI5Event) {
    const listItem = event.getParameter("listItem") as ColumnListItem;
    const context = listItem.getBindingContext() as Context;
    context.delete();
  }

  public sayHello(): void {
    MessageBox.show("Hello World!");
  }

  public mySearchProduct(event:UI5Event){
	const query = event.getParameter("query");
	MessageBox.show(query);
	const table = this.getView().byId("productsList");
	const binding = table.getBinding("items") as ODataListBinding;
	const filters = [];
	if(query){
		filters.push(new Filter("title", FilterOperator.Contains, query))
	}
	binding.filter(filters);
  }

  public onItemPress(event: UI5Event){
    var item = event.getSource();
    var context = item.getBindingContext();
    var path = context.getPath().substr(1);
    this.navTo("detail", {path: path});
  }

  public onScanNavi(){
    var router = this.getOwnerComponent().getRouter();
    router.navTo("barcode");
  }
}
\end{lstlisting}
\end{mdframed}

\begin{mdframed}[backgroundcolor=mygrey2, leftmargin=0.5cm, hidealllines=true, innerleftmargin=3pt, innerrightmargin=0cm, innertopmargin=0cm, innerbottommargin=-3cm, splitbottomskip=0]
\begin{lstlisting}[emph={event, UI5Event, listItem, detail, path}, caption=Detail Controller der Anwendung]
import BaseController from "./BaseController";
import UI5Event from "sap/ui/base/Event";

/**
 * @namespace tff.Products.controller
 */
export default class App extends BaseController {

	public onInit() : void {
		var route = this.getRouter();
		route.getRoute("detail").attachPatternMatched(this._onRouteMatched,this);
	}
	public _onRouteMatched(event: UI5Event) {
		var path = event.getParameter("arguments"). path;
		this.getView().bindElement({path:"/" + path});
	}
	public onNaviBack() {
		this.onNavBack();
	}
}
\end{lstlisting}
\end{mdframed}

\begin{mdframed}[backgroundcolor=mygrey2, leftmargin=0.5cm, hidealllines=true, innerleftmargin=3pt, innerrightmargin=0cm, innertopmargin=0cm, innerbottommargin=-3cm, splitbottomskip=0]
\begin{lstlisting}[emph={event, UI5Event, listItem, detail, path}, caption=Base Controller der Anwendung]
import Controller from "sap/ui/core/mvc/Controller";
import UIComponent from "sap/ui/core/UIComponent";
import AppComponent from "../Component";
import Model from "sap/ui/model/Model";
import ResourceModel from "sap/ui/model/resource/ResourceModel";
import ResourceBundle from "sap/base/i18n/ResourceBundle";
import Router from "sap/ui/core/routing/Router";
import History from "sap/ui/core/routing/History";

/**
 * @namespace tff.Products.controller
 */
export default abstract class BaseController extends Controller {

	/**
	 * Convenience method for accessing the component of the controller's view.
	 * @returns The component of the controller's view
	 */
	public getOwnerComponent(): AppComponent {
		return (super.getOwnerComponent() as AppComponent);
	}

	/**
	 * Convenience method to get the components' router instance.
	 * @returns The router instance
	 */
	public getRouter() : Router {
		return UIComponent.getRouterFor(this);
	}

	/**
	 * Convenience method for getting the i18n resource bundle of the component.
	 * @returns The i18n resource bundle of the component
	 */
	public getResourceBundle(): ResourceBundle | Promise<ResourceBundle> {
		const oModel = this.getOwnerComponent().getModel("i18n") as ResourceModel;
		return oModel.getResourceBundle();
	}

	/**
	 * Convenience method for getting the view model by name in every controller of the application.
	 * @param [sName] The model name
	 * @returns The model instance
	 */
	public getModel(sName?: string) : Model {
		return this.getView().getModel(sName);
	}

	/**
	 * Convenience method for setting the view model in every controller of the application.
	 * @param oModel The model instance
	 * @param [sName] The model name
	 * @returns The current base controller instance
	 */
	public setModel(oModel: Model, sName?: string) : BaseController {
		this.getView().setModel(oModel, sName);
		return this;
	}

	/**
	 * Convenience method for triggering the navigation to a specific target.
	 * @public
	 * @param sName Target name
	 * @param [oParameters] Navigation parameters
	 * @param [bReplace] Defines if the hash should be replaced (no browser history entry) or set (browser history entry)
	 */
	public navTo(sName: string, oParameters?: object, bReplace?: boolean) : void {
		this.getRouter().navTo(sName, oParameters, undefined, bReplace);
	}

	/**
	 * Convenience event handler for navigating back.
	 * It there is a history entry we go one step back in the browser history
	 * If not, it will replace the current entry of the browser history with the main route.
	 */
	public onNavBack(): void {
		const sPreviousHash = History.getInstance().getPreviousHash();
		if (sPreviousHash !== undefined) {
			window.history.go(-1);
		} else {
			this.getRouter().navTo("main", {}, undefined, true);
		}
	}

}
\end{lstlisting}
\end{mdframed}


\begin{mdframed}[backgroundcolor=mygrey2, leftmargin=0.5cm, hidealllines=true, innerleftmargin=3pt, innerrightmargin=0cm, innertopmargin=0cm, innerbottommargin=-3cm, splitbottomskip=0]
\begin{lstlisting}[emph={event, UI5Event, listItem, detail, path}, caption=Barcode Controller der Anwendung]
/**
 * @namespace tff.Products.controller
 */
export default class Main extends BaseController {
	private formatter = formatter;
	private oScanResultText;

	public onScanSuccess(oEvent) : void {
		if (oEvent.getParameter("cancelled")) {
			MessageToast.show("Scan cancelled", { duration:1000 });
		} else {
			if (oEvent.getParameter("text")) {
				oScanResultText.setText(oEvent.getParameter("text"));
			} else {
				oScanResultText.setText('');
			}
		}
	}

	public onScanError(oEvent) : void {
		MessageToast.show("Scan failed: " + oEvent, { duration:1000 });
	}

	public onAfterRendering() : void {
		oScanResultText = this.getView().byId("result");
		var oScanButton = this.getView().byId("button");
		if (oScanButton) {
			$(oScanButton.getDomRef()).on("click", function(){
				oScanResultText.setText('');
			});
		}
	}
}
\end{lstlisting}
\end{mdframed}



\begin{mdframed}[backgroundcolor=mygrey2, leftmargin=0.5cm, hidealllines=true, innerleftmargin=3pt, innerrightmargin=0cm, innertopmargin=0cm, innerbottommargin=-3cm, splitbottomskip=0]
\begin{lstlisting}[emph={event, UI5Event, listItem, detail, path}, caption=App Controller der Anwendung]
import BaseController from "./BaseController";
/**
 * @namespace tff.Products.controller
 */
export default class App extends BaseController {
	public onInit() : void {
		// apply content density mode to root view
		this.getView().addStyleClass(this.getOwnerComponent().getContentDensityClass());
	}
}
\end{lstlisting}
\end{mdframed}

























