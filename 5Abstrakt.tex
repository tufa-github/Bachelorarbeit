%%%%%%%%%%%%%%%%%%%%%%%%%%%%%%%%%%%%%%%%%%%%%%%%%%%%%%%%%%%%%%%%%
%        Contents: Bachelorarbeit, HS Fulda        %
%                          31.08.2022                        %
%---------------------------------------------------------%
%                           Abstrakt.tex                      %
%                        by Fangfang Tan                   %
%         fangfang.tan@informatik.hs-fulda.de       %
%%%%%%%%%%%%%%%%%%%%%%%%%%%%%%%%%%%%%%%%%%%%%%%%%%%%%%%%%%%%%%%%%

{\color{white} h}\\
\vspace{0.5cm}

\begin{abstract}% deutsche Fassung
\thispagestyle{plain}
Die vorliegende Bachelorarbeit befasst sich mit der Low-Code Anwendungsentwicklung mit SAP AppGyver im Vergleich zur nativen SAP Fiori-Entwicklung, wobei der Fokus auf drei Technologien liegt: SAP AppGyver, Fiori Elements und SAPUI5. Anhand eines kundenorientierten Anwendungsfalles werden Anwendungen mit den drei Technologien entwickelt, um die drei Technologien zu betrachten und zu vergleichen. Eine Reihe von Funktionalitäten außerhalb des Anwendungsfalls wird ebenfalls untersucht, damit die drei Technologien eingehend bewertet werden können. Anschließend werden Bewertungsmatrizen definiert, Bewertungen durchgeführt und schließlich die Bewertungsergebnisse interpretiert und diskutiert. Mit Hilfe der Bewertungsergebnisse werden die Vor- und Nachteile der Technologie ermittelt.
\end{abstract}

%{\color{white} h}\\
\vspace{2cm}

\renewcommand{\abstractname}{Abstract}
\begin{abstract}% english version
\thispagestyle{plain}
This bachelor thesis is dealing with low-code application development with SAP AppGyver compared to native SAP Fiori development, with the focus on three technologies: SAP AppGyver, Fiori Elements and SAPUI5. A customer-oriented use case will be used to develop applications with the three technologies for consideration and comparison. A set of functionalities outside of the use case will also be studied so that the three technologies can be evaluated in more depth. Then, evaluation matrices will be defined, evaluations will be carried out, and finally the evaluation results will be interpreted and discussed. With the evaluation results, the advantages and disadvantages of the technologies will be identified.
\end{abstract}
